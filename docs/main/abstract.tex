% background
Two required inputs to mathematical models of sexually transmitted infections are
the average duration in epidemiological risk states (\eg selling sex) and
the average rates of sexual partnership change.
These variables are often only available as aggregate estimates from
published cross-sectional studies, and may be subject to
distributional, sampling, censoring, and measurement biases.
% methods
We explore adjustments for these biases using aggregate estimates of
duration in sex work and numbers of reported sexual partners
from a published 2011 survery of female sex worker in Eswatini.
We develop adjustments from first principles,
and construct Bayesian hierarchical models to reflect
our mechanistic assumptions about the bias-generating processes.
% results
We show that different mechanisms of bias for duration in sex work may
``cancel out'' by acting in opposite directions,
but that failure to consider some mechanisms could over- or underestimate
duration in sex work by factors approaching 2.
We also show that conventional interpretations of sexual partner numbers
are biased due to implicit assumptions about partnership duration,
but that unbiased estimators of partnership change rate can be defined
that explicitly incorporate a given partnership duration.
We highlight how the unbiased estimator is most important when
the survey recall period and partnership duration are similar in length.
% conclusions
While we explore these bias adjustments using a particular dataset,
and in the context of deriving inputs for mathematical modelling,
we expect that our approach and insights would be applicable to
other datasets and motivations for quantifying sexual behaviour data.
% SM: nice - i think ok to be general/broad in abstract,
%     but discussion could benefit from 1-2 lines re: concrete examples
%     e.g. factors associated with HIV incidence/prevalence, etc.?
% TODO
