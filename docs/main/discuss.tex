\section{Discussion}
We sought to develop bias adjustments for estimating
the mean duration in epidemiological risk states (or periods of risk)
and rates of sexual partnership change
from aggregate cross-sectional data.
We developed these adjustments using Bayesian hierarchical models to incorporate
uncertainty in the available data and mechanistic assumptions about
several ``hidden'' bias-generating processes.
We showed that these adjustments can influence estimated variable means by factors approaching 2,
suggesting that unadjusted estimates of these variables should be interpreted carefully.
\par
We grounded our study in the analysis of aggregate sex work data
to parameterize a mathematical model of HIV transmission.
However, our approach should be broadly applicable to
analysis of other intermittent risk exposures and event rates,
including analysis of individual-level data for conventional statistical models.
For example, periods of hazardous conditions may need to be quantified
in an empiric study of workplace injury risk.
Additionally, estimates of population-attributable fractions
may be improved through our insight that: in some cross-sectional studies,
reported exposure duration reflects only half of the total expected exposure duration.
\par
Our work can also be built upon by considering
further potential sources of bias and/or uncertainty.
For example, we assumed a fixed duration for each sexual partnership type,
but this duration could be modelled as another random variable
whose distribution could also be inferred.
Moreover, future work could consider
rounding error~\cite{Mills2014},
recall bias~\cite{Ramjee1999},
reporting bias~\cite{Lowndes2012},
and the like \cite{Fenton2001}.
% SM: I love this section above re: future work.
%     Do most of these papers end with that, or do we want to end with the "broadly applicable" part?
% JK: Good point -- I tried them flipped but I think prefer this way.
%     we can see if reviewers have a preference
