\section{Introduction}
% SM: this is good! after you revise, can I look at the intro & discussion once agaiin quickly before we submit? %SM: updated intro is great!
Epidemic modelling of sexually transmitted infections (STI) relies on
quantification of sexual behaviour for model inputs (parameters) \cite{Garnett2002}.
In models of STI transmission with risk heterogeneity
--- \ie considering subgroups that experience differential risks ---
two important parameters are:
the duration of time within a ``risk group'' (more recently referrred to as season of risk in HIV/STI epidemiology); %SM: check if citation to the latter, but in most HIV meetings, people are using this term now :) - so just thinking to relate it easier for uptake of this paper for others
and the rate of sexual partnership change (often stratified by partnership type) %SM: maybe parenthesis to denote that stratification refers to the latter and not t he former? 
\cite{Garnett1996,Stigum1997,Watts2010,Knight2020}.
For example, the average duration of time engaged in sex work
can be used to define the modelled rate of ``turnover'' among sex workers \cite{Watts2010}.
Similarly, the numbers of main, casual, transactional, and/or paying sexual partners per year
can be used to define the modelled rate of infection incidence \cite{Boily2015}.
\par
Data to inform these parameters largely come from cross-sectional studies,
and are often only available as aggregate estimates (\vs individual-level data).
Such estimates may be subject to distributional, sampling, censoring, and measurement biases.
Our aim is therefore to explore bias adjuments for estimating: %SM: nice
(1) duration in a risk group, and (2) rate of partnership change,
from aggregate cross-sectional survey data, considering these factors. %SM: which factors?
We explore these topics using data from %SM: explore or demonstrate?
a 2011 female sex worker survey in Eswatini~\cite{Baral2014},
in order to support parameterization of a heterosexual HIV transmission model.
