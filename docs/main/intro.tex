\section{Introduction}
Quantifying sexual behaviour
is necessary to study the epidemiology of sexually transmitted infections (STI),
including to inform inputs for STI transmission modelling \cite{Fenton2001}.
Two important quantities are:
the duration of time within a ``risk group'' such as female sex workers (FSW), and
the rate of new partnership formation, possibly stratified by partnership type.
\par
Our aims are to motivate and discuss bias adjuments for estimating:
\begin{enumerate}
  \item duration in a risk group
  \item rate of partnership change
\end{enumerate}
from cross-sectional survey data,
considering issues of sampling bias and censoring.
We explore these topics using aggregate data from
two female sex worker (FSW) surveys in Eswatini:
\cite{Baral2014}~(2011, RDS sampling, $N = 328$) and
\cite{EswKP2014}~(2014, PLACE sampling, $N = 781$).
